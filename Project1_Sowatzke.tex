\documentclass[conference]{IEEEtran}
\IEEEoverridecommandlockouts
\usepackage{matlab-prettifier}
\usepackage{cite}
\usepackage{amsmath,amssymb,amsfonts,nccmath}
\usepackage{algorithmic}
\usepackage{graphicx}
\usepackage{textcomp}
\usepackage{xcolor}
\usepackage{float}
\usepackage{tabularx}
\def\BibTeX{{\rm B\kern-.05em{\sc i\kern-.025em b}\kern-.08em
    T\kern-.1667em\lower.7ex\hbox{E}\kern-.125emX}}

%\renewenvironment{align*}{\par$\!\aligned}{\endaligned$\par}

%\setlength{\mathindent}{1cm}

\begin{document}

\title{Project 1 : Matched Filtering of an LFM Waveform}

\author{\IEEEauthorblockN{Owen Sowatzke}
\IEEEauthorblockA{\textit{Electrical Engineering Department} \\
\textit{University of Arizona}\\
Tucson, USA \\
osowatzke@arizona.edu}}
\maketitle

\begin{abstract}
The range resolution of a rectangular pulse is inversely proportional to the length of the rectangular pulse. Therefore, to achieve fine range resolution, a short rectangular pulse is required. However, decreasing the pulse length also leads to a reduction in SNR. Pulse compression waveform seek to decouple pulse length from range resolution and in turn make it possible to simultaneously achieve fine range resolution and a high SNR. One common pulse compression waveform is a linear frequency modulated (LFM) waveform. This document simulates the matched filer response of an LFM waveform and examines the resulting SNR and range resolution.
\end{abstract}

\begin{IEEEkeywords}
Pulse Compression, Linear Frequency Modulated (LFM) Waveform, Matched Filter
\end{IEEEkeywords}
\section{Introduction}
A linear frequency modulated (LFM) waveform is a pulse compression waveform defined according to the following equation:
\begin{equation}
x(t)=e^{j\pi\beta t^2/\tau}
\end{equation}
where:
\begin{fleqn}[\parindent]
\begin{align*}
\beta &= \text{Waveform Bandwidth (Hz)}\\
\tau &= \text{Waveform Duration (s)}
\end{align*}
\end{fleqn}
A sample pulsed radar system will be created in MATLAB to simulate the matched filter response of this waveform. Using the radar system model, the increase in SNR due to the matched filter will be examined. Additionally, the return of multiple targets will be simulated to determine the system's range resolution.
\section{Sample Radar System Parameters}
The pulse radar system simulated in MATLAB was designed with the following set of parameters: 
\begin{table}[H]
\caption{Radar System Parameters}
\label{Parameter Table}
\begin{tabularx}{0.5\textwidth}{| X | X |}
\hline
Carrier Frequency & 10 GHz \\
\hline
Sample Rate & 100 MHz \\
\hline
Transmit Power & 20 dB \\
\hline
Antenna Gain & 44.15 dB \\
\hline
Noise Figure & 10 dB \\
\hline 
System Losses & 5 dB \\
\hline
PRF & 50 kHz \\
\hline
Duty Cycle & 20\% \\
\hline
Chirp Bandwidth & 100 MHz \\
\hline
\end{tabularx}
\end{table}
\noindent
Additional parameters can be computed using the parameters given in Table \ref{Parameter Table}. For example, the radar system's PRI can be computed as follows:
\begin{equation}
PRI = \frac{1}{PRF} = 20 {\mu}s
\label{PRI Equation}
\end{equation}
Another parameter of interest, the length of the LFM pulse ($\tau$), can be computed using the duty cycle ($D$) as follows:
\begin{equation}
\tau = PRI \cdot D = 4 {\mu}s
\label{tau equation}
\end{equation}
%\section{Introduction}
%Both Monte Carlo (MC) Simulations and Importance Sampling (IS) are techniques for estimating probabilities. In this document, both techniques will be used to estimate the probability that a random variable $X$ with distribution $X\sim N(1,1)$ exceeds 3.957 (i.e. $P\{X > 3.957\}$).
%\section{Monte Carlo Simulation}
%The probability of interest can be written as follows:
%\begin{equation}
%P\{X > 3.957\} = \int_{-\infty}^{\infty}I(x)f_X(x)dx
%\label{Pint}
%\end{equation}
%where
%\begin{equation}
%I(x) = \begin{cases}
%1 & x > 3.957 \\
%0 & x \leq 3.957
%\end{cases}
%\label{Ix_def}
%\end{equation}
%Note that this probability can also be written as an expected value:
%\begin{equation}
%P\{X > 3.957\} = E[I(X)]
%\end{equation}
%Monte Carlo Simulation attempts to estimate this probability using a sample mean
%\begin{equation}
%P\{X > 3.957\} \approx \frac{1}{N}\sum_{i=1}^{N}I(x_i)
%\end{equation}
%where
%\begin{equation}
%x_1, x_2, ..., x_N \sim f_X(x)
%\end{equation}
%\par
%Given different values of $N \in [1, 10^5]$, Monte Carlo Simulations can be used to estimate $P\{X > 3.957\}$. This estimate can than be compared to the true probability.
%\begin{equation}
%P\{X > 3.957\} = 0.001553
%\end{equation}
%The Monte Carlo probability estimate was generated for values $N$ from $100$ to $10^5$ in increments of $100$. The MATLAB code to generate this estimate is included in Appendix \ref{matlab_code}. The Monte Carlo probability estimate is plotted vs $N$ in Fig. \ref{Monte Carlo Estimate}
%\begin{figure}[H]
%\centerline{\includegraphics[width=0.5\textwidth]{Monte_Carlo_Estimate.png}}
%\caption{Monte Carlo Probability Estimate}
%\label{Monte Carlo Estimate}
%\end{figure}
%\noindent
%The absolute value of the error between the Monte Carlo estimate and the true probability ($|P_{est}\{X > 3.957\} - P_{truth}\{X > 3.957\}|$) is also plotted vs $N$.
%\begin{figure}[H]
%\centerline{\includegraphics[width=0.5\textwidth]{Monte_Carlo_Error.png}}
%\caption{Monte Carlo Probability Error}
%\label{Monte Carlo Perr}
%\end{figure}
%As $N\rightarrow\infty$, the Monte Carlo Estimate should approach the true probability. This is confirmed by the reduction in variance as $N$ increases in Fig. \ref{Monte Carlo Perr}.
%\section{Importance Sampling}
%Examining the shaded area in Fig. 
%\ref{Monte Carlo Limits}, very few of the samples used in the Monte Carlo estimate actually lie within the shaded region. This  leads to large variances in the Monte Carlo probability estimates.
%\begin{figure}[H]
%\centerline{\includegraphics[width=0.5\textwidth]{Monte_Carlo_Limitations.png}}
%\caption{Computing Probability from $f_X(x)$}
%\label{Monte Carlo Limits}
%\end{figure}
%Importance Sampling attempts to overcome these limitations. It re-expresses equation \eqref{Pint} in the following form:
%\begin{equation}
%P\{X > 3.957\} = \int_{-\infty}^{\infty}I(x)f_X(x)\frac{g_X(x)}{g_X(x)}dx
%\label{is_int}
%\end{equation}
%where $I(x)$ is given by equation \eqref{Ix_def}.
%Equation \eqref{is_int} can be written as follows:
%\begin{equation}
%P\{X > 3.957\} = \int_{-\infty}^{\infty}I(x)\frac{f_X(x)}{g_X(x)}g_X(x)dx
%\end{equation}
%This equation can also be written as an expected value.
%\begin{equation}
%P\{X > 3.957\} = E\left[I(x)\frac{f_X(x)}{g_X(x)}\right]
%\end{equation}
%Once again, the expected value can be estimated using sample mean.
%\begin{equation}
%P\{X > 3.957\} \approx \frac{1}{N}\sum_{i=1}^{N}I(x_i)\frac{f_X(x_i)}{g_X(x_i)}
%\end{equation}
%Note the samples $x_1,x_2,...,x_N$ are sampled from $g_X(x)$ instead of $f_X(x)$. This enables us to select a density $g_X(x)$ with more "important" samples. 
%\par
%Consider a density $g_X(x)$ that is a shifted version of $f_X(x)$. Specifically, consider the following density function
%\begin{equation}
%g_X(x) = \frac{1}{\sqrt{2\pi}}e^{-(x-3.957)^2/2}
%\end{equation}
%The important samples of this density function are highlighted in Fig. \ref{Important Samples}. 
%\begin{figure}[H]
%\centerline{\includegraphics[width=0.5\textwidth]{Important_Samples.png}}
%\caption{Important Samples from $g_X(x)$}
%\label{Important Samples}
%\end{figure}
%\noindent
%Note that the number of important samples is significantly larger than that of Fig. \ref{Monte Carlo Limits}. This should reduce the variance of the probability estimate. 
%\par
%The Importance Sampling probability estimate was generated for values $N$ from $100$ to $10^5$ in increments of $100$. The MATLAB code to generate this estimate is included in Appendix \ref{matlab_code}. The Important Sampling probability estimate is plotted vs $N$ in Fig. \ref{Importance Sampling Estimate}.
%\begin{figure}[H]
%\centerline{\includegraphics[width=0.5\textwidth]{Importance_Sampling_Estimate.png}}
%\caption{Importance Sampling Probability Estimate}
%\label{Importance Sampling Estimate}
%\end{figure}
%\noindent
%The absolute value of the error between the Importance Sampling estimate and the true probability ($|P_{est}\{X > 3.957\} - P_{truth}\{X > 3.957\}|$) is also plotted vs $N$.
%\begin{figure}[H]
%\centerline{\includegraphics[width=0.5\textwidth]{Importance_Sampling_Error.png}}
%\caption{Importance Sampling Probability Error}
%\label{Importance Sampling Perr}
%\end{figure}
%The variance of the Monte Carlo and Importance Sampling probability estimates can be compared by overlaying the probability error plots. This plot is shown in Fig. \ref{Error Overlay}
%\begin{figure}[H]
%\centerline{\includegraphics[width=0.5\textwidth]{Error_Overlay.png}}
%\caption{Probability of Error Comparison}
%\label{Error Overlay}
%\end{figure}
%\noindent
%Note that the Importance Sampling probability estimate is much closer to the expected probability than the Monte Carlo Estimate. As such, the Importance Sampling estimate has a much lower variance for the same values of $N$.
%\section{Conclusion}
%Monte Carlo Simulation and Importance Sampling are ways to estimate probabilities. Both of these approaches were used to estimate $P\{X > 3.957\}$ for $X \sim N(1,1)$. As demonstrated, importance sampling produced a probability estimate with a smaller variance than that of Monte Carlo simulation. This result can be applied to other probabilities estimates as well. If a distribution is hard to sample, importance sampling can be used to reduce the number of samples required to produce an estimate. This can make a significant difference, especially when it is computationally intensive to determine if the random variable exceeds the threshold.
%\onecolumn
%\pagebreak
%\appendices
%\section{MATLAB Source Code}
%\label{matlab_code}
%\lstset{style=Matlab-editor}
%\lstinputlisting{Project3_Sowatzke.m}
%\raggedbottom
\end{document}
