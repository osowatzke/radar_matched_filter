\documentclass[12pt,letterpaper]{article}
\usepackage[margin=1in]{geometry}
\usepackage{tabularx}
\usepackage{float}
\usepackage{graphicx}
\title{Project 1}
\author{Owen Sowatzke}
\date{October 10, 2022}

\begin{document}
\maketitle
\section{LFM Parameters:}
A pulsed radar system was generated in MATLAB using the following parameters:
\begin{table}[H]
\caption{Radar System Parameters}
\label{Parameter Table}
\begin{tabularx}{\textwidth}{| X | X |}
\hline
Carrier Frequency & 10 GHz \\
\hline
Sample Rate & 100 MHz \\
\hline
Transmit Power & 20 dB \\
\hline
Antenna Gain & 44.15 dB \\
\hline
Noise Figure & 10 dB \\
\hline 
System Losses & 5 dB \\
\hline
PRF & 50 kHz \\
\hline
Duty Cycle & 20\% \\
\hline
Chirp Bandwidth & 100 MHz \\
\hline
\end{tabularx}
\end{table}
\noindent
Using the parameters given in Table \ref{Parameter Table}, the radar system's PRI can be computed as follows:
\begin{equation}
PRI = \frac{1}{PRF} = 20 {\mu}s
\label{PRI Equation}
\end{equation}
Let the radar system's duty cycle be given by $D$. Then, the length of the LFM pulse ($\tau$) is given as follows:
\begin{equation}
\tau = PRI \cdot D = 4 {\mu}s
\label{tau equation}
\end{equation}
\section{System Modeling}
To simulate the given radar system, the received signal for each of the target returns must be generated. This received signal is then passed through a matched filter to increase the SNR of the target returns.
\subsection{Modeling the Received Signal}
The received signal is generated according to the block diagram given in Fig. \ref{Generate RX Sig}.
\begin{figure}[H]
\center{\fbox{\includegraphics[width=0.5\textwidth]{gen_rx_sig.png}}}
\caption{Generating Received Signal}
\label{Generate RX Sig}
\end{figure}
\noindent
The transmitted signal is an LFM waveform padded with zeros to length of the pulse. The LFM waveform is generated using the following formula:
\begin{equation}
x(t) = e^{j\pi \beta t^2/\tau}
\end{equation}
where $\beta$ is the bandwidth of the chirp waveform and $\tau$ is the length of the chirp waveform. The received signal is generated using a scaled and delayed versions of the transmitted signal. The delay of each target can be computed using the following formula:
\begin{equation}
t_d = \frac{2R}{c}
\end{equation}
Each of the target returns is then scaled to provide an SNR given by the following formula:
\begin{equation}
SNR = \frac{P_t G^2 \lambda^2 \sigma}{(4\pi)^3 R^4 k T_0 B_n F_n L_s L_\alpha(R)}
\end{equation}
The noise has unit power. Therefore, to provide the required SNR, each of the target returns must be scaled by $10^{SNR/20}$. The received signal is the sum of each of the target returns and noise. Assuming the radar system has a duplexer, the received signal will also have blanking (i.e. during transmission the receiver will not receive any signal). To model this blanking, the first $L$ samples of the received signal are replaced with zeros.
\subsection{Modeling the Matched Filter}
The matched filter output can be generated according to the block diagram shown in Fig. \ref{Generate MF Output}.
\begin{figure}[H]
\center{\fbox{\includegraphics[width=0.2\textwidth]{gen_mf_output.png}}}
\caption{Generating Matched Filter Output}
\label{Generate MF Output}
\end{figure}
\section{Matched Filter SNR}
The SNR of a target return can be examined at the ADC and after the matched filter. The SNR of the matched filter output should by larger than the SNR of the ADC input by a factor of the filter length. This result will be confirmed via MATLAB simulation.
\subsection{SNR of the Received Signal}
The transmitted LFM signal can be generated using to the following formula:
\begin{equation}
x(t) = e^{j\pi \beta t^2/\tau}
\end{equation}
The transmitted LFM signal is padded with zeros to achieve the required pulse length.

\noindent
The received signal contains both Gaussian noise and a scaled and delayed version of the transmit signal. The SNR of the received signal is given by the following formula:
\begin{equation}
SNR = 10log_{10}\frac{P_s}{P_n}
\end{equation}
The transmitted LFM pulse is delayed and scaled to generate the received signal.

The received signal can be generated by delaying the transmitted LFM pulse. This delayed pulse is then scaled to provide a given signal power.
The SNR of the received signal is given by
\begin{equation}
\chi = \frac{P_t G^2 \lambda^2 \sigma}{(4\pi)^3 R^4 k T_0 B_n F_n L_s L_\alpha(R)}
\end{equation}

\end{document}
